\documentclass{article}



\usepackage{arxiv}

\usepackage[utf8]{inputenc} % allow utf-8 input
\usepackage[T1]{fontenc}    % use 8-bit T1 fonts
\usepackage{hyperref}       % hyperlinks
\usepackage{url}            % simple URL typesetting
\usepackage{booktabs}       % professional-quality tables
\usepackage{amsfonts}       % blackboard math symbols
\usepackage{nicefrac}       % compact symbols for 1/2, etc.
\usepackage{microtype}      % microtypography
\usepackage{lipsum}		% Can be removed after putting your text content
\usepackage{graphicx}
\usepackage[super]{natbib}
\usepackage{doi}
\usepackage{float}
\usepackage{amsmath}
\usepackage{mathrsfs}
\usepackage{caption} 
\usepackage{multicol}
\usepackage{pgfplots}
\usepackage{sectsty}
\usepackage{cancel}
\usepackage{amsmath}
\usepackage{amssymb}
\usepackage{listings}
\usepackage{algorithm}
\usepackage{algpseudocode}
\DeclareUnicodeCharacter{2212}{-}
\usetikzlibrary{patterns}
\usetikzlibrary{calc}


\title{\emph{Randomized Algorithm}}

\newcommand{\mycomment}[1]{}
\NewDocumentCommand{\codeword}{v}{%
\texttt{\textcolor{blue}{#1}}%
}


%\date{September 9, 1985}	% Here you can change the date presented in the paper title
%\date{} 					% Or removing it

\author{{\hspace{1mm}Rajat Dua} \\
	Master of Science - Computer Science\\
	Aarhus University\\
	\texttt{au747653@uni.au.dk / 202303549@post.au.dk} \\
	%% examples of more authors
	\And
	{\hspace{1mm}Ioannis Caragiannis} \\
	Institut for Datalogi\\
	Aarhus University\\
	%% \AND
	%% Coauthor \\
	%% Affiliation \\
	%% Address \\
	%% \texttt{email} \\
	%% \And
	%% Coauthor \\
	%% Affiliation \\
	%% Address \\
	%% \texttt{email} \\
	%% \And
	%% Coauthor \\
	%% Affiliation \\
	%% Address \\
	%% \texttt{email} \\
}

% Uncomment to remove the date
\date{January 29, 2024}

% Uncomment to override  the `A preprint' in the header
\renewcommand{\headeright}{Rajat Dua}
\renewcommand{\undertitle}{Week 5 - Introduction}
\renewcommand{\shorttitle}{\textit{Randomized Algorithm} Notes}

%%% Add PDF metadata to help others organize their library
%%% Once the PDF is generated, you can check the metadata with
%%% $ pdfinfo template.pdf
\hypersetup{
pdftitle={Randomized Algorithm},
pdfsubject={q-bio.NC, q-bio.QM},
pdfauthor={Rajat Dua},
pdfkeywords={randomized, notes, au, aarhus, university},
}
\captionsetup[table]{skip=10pt}

% \subsectionfont{\underline}


\begin{document}
\maketitle
\vspace{-1cm}
\section{Introduction}
\hrule
\hrule

\vspace{1cm}

\subsection{Usual assumptions}

\begin{itemize}
    \item Fair coin $P_r[heads] = P_r[tails] = \frac{1}{2}$
    \item Random selection among a finite set of items
    \item Access to a random permutation of elements
    \item Selection of a random point in the interval $[0,1]$
    \item Selection from a finite or infinite set according to a non-uniform probability distribution
\end{itemize}

\subsection{Main characteristics}

\begin{enumerate}
    \item Deterministic: Same input, same output
    \item Randomized: Same input, results into same or different output
    \item The execution time is the same for deterministic algorithms
\end{enumerate}

\subsection{Why do we need it?}
Maybe we want a solution that works well on "\textbf{average}". Or at least it should create a "\textbf{high-probability}" solution.

\textbf{Derandomization}: we can work our way backwards.

\subsection{Examples}

\subsubsection{Contention Resolution}

Transmitting a message between two nodes, if a collision happens when two messages are sent (simultaneously) the protocol says you need to re-transmit the message.

There are defined protocols with an amazing timing system that avoids collisions.

\subsubsection{Computing a large cut in a graph}

Partition a set of nodes in two subsets of nodes, the edges need to be maximised between the two disjointed subsets.

Just be greedy. You can count the edges from an assigned colour to a node. There are some maximization decisions when there isn't one, it makes an "arbitrary" choice too.

This is an NP-hard problem.

Can we do it better? How good is the algorithm?

Let's just introduce "\textbf{randomness}". A red/Blue colour is assigned for each node.

$X_e$ is the random variable that denotes whether $e$ is part of the "cut".

Two nodes can have possibly 4 options - red-blue, blue-red, red-red, or blue-blue. It means the probability is 1/4. However, we need a distinct count so the probability is 1/2.

\subsubsection{Formal representation for large cut example}

$\mathcal{C}$ = Size of cut returned by algorithm.

$\mathbb E$ = Expectation

$\mathbb E[C] = E[\sum_{e\in E} X_e] \geq Opt/2$

\textbf{Terms}
\begin{itemize}
    \item Linearity of Expectation
    \item Concentration Bounds
    \item Markov Inequality
\end{itemize}

Repeated execution of the same "randomized" algorithm gives a "good" solution.

\section{Quicksort}

A sorting algorithm that uses pivot and recursion which reorganizes the elements around the "pivot".

You need to split the partitions in equal size to make it "optimal"

For the "first try" algorithm which uses first index as the pivot can have the worst case scenario: $O(n^2)$ where the list is reverse sorted.

The selection of the pivot is \textbf{crucial}. To make it from $O(n^2)$ to $O(nlogn)$ you can select the pivot on the basis of "median".

$T(n/2)$ = Dividing the problem into two halves.

$T_{parition}(n)$ = Time taken for partition, there is a linear algorithm O(n)

$T_{median}(n)$

> Master Theorem 

\textbf{Tree structure} 

$T(n) [O(n)] \rightarrow T(n/2), T(n/2) [O(n)]$

$T(n/2) \rightarrow T(n/4), T(n/4) \ldots log(n)$

\section{Randomized Quicksort}

We need to exploit the median finding and reach $O(nlogn)$. The probability of the worst case when we are trying to pick a number at random will reach $1/n!$

On average, a random split gives us a 25-75 split.

$T(n) = T(3n/4) + T(n/4) + O(n)$

$3n/4 = 75\%$ Split

$n/4 = 25\%$ Split

We can create the same master theorem tree here as well.

$log_{\frac{4}{3}}(n) = \frac{log \: n}{log \frac{4}{3}}$

This doesn't make a lot of sense but the idea is great to build on it.

<Insert page from notes app> :: It talks about the maths behind $o(log n)$ average for a coin toss.

\subsection{How to do comparisons?}

The comparison between the pivot and the element from left or right to put it in the next.

We need to get rid of some pair of comparisons which happen in $O(n^2)$ and we should not reach the worst-case scenario.

Given input array, $z_j$, $i$-th smallest element.

$X_{ij}$ is the r.v (random variable) that compares between $z_i$ and $z_j$ $(i<j)$ for the indices from $\{1,2 \ldots n\}$

$X_{ij} = 1$ if $z_i$ and $z_j$ are in the same subarray. Otherwise, it is 0

If the numbers are compared once that's good enough. You don't have to do it again.

<Insert page from the notes app>: diagram of pivots and comparisons

\subsubsection{Bounding the expectation of $C$}

$\mathbb E[C] = \mathbb E[\sum^{n-1}_{i=1} \sum_{j=i+1}^{n} X_{ij}] = \sum^{n-1}_{i=1} \sum_{j=i+1}^{n} \mathbb E[X_{ij}] = \sum^{n-1}_{i=1} \sum_{j=i+1}^{n} P_r[X_{ij} = 1]$

\subsubsection{Cases for pivots}

$z_2$ as pivot $P_r = 1/n$, no comparisons
$z_n$ as pivot $P_r = 1/n$, no comparisons
If in between $z_3$ to $z_{n-1}$ no comparisons in between

if $z_1$ is selected, $z_2$ and $z_n$ will be in the same subarray. If they are compared in the next phase, there will be some probability that they will be compared again or not.

For the case in between, what is the probability that they will be compared? 

We can ignore the elements smaller than $z_i$ and larger than $z_j$. So the total number of elements is $j-i+1$ and there are two cases to select from either $z_i$ or $z_j$ so the overall probability will be - 

$P_r[X_{ij} = 1] = \frac{2}{j-i+1}$

The proof has some issues with the summation bounds.

$H_n$ - Harmonic of $n = log n = \sum^{n}_{k=1} \frac{1}{k}$

<Insert page from notes app>: proof of "utmost" bound instead of least as mentioned in the slide.

\bibliographystyle{unsrtnat}
\bibliography{references}

\end{document}
